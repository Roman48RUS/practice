\documentclass[a4paper]{report}
\usepackage[14pt]{extsizes}
\usepackage[utf8]{inputenc}
\usepackage[english,russian]{babel}
\usepackage[OT1]{fontenc}
\usepackage{setspace, amsmath}
\usepackage{amsfonts}
\usepackage{amssymb}
\usepackage[left=20mm, top=10mm, right=15mm, bottom=20mm, nohead, footskip=10mm]{geometry}
\usepackage{gensymb}

\usepackage{graphicx} 
\graphicspath{{images/}}
\DeclareGraphicsExtensions{.pdf,.png,.jpg}

\renewcommand{\theenumi}{\arabic{enumi}}
\renewcommand{\labelenumi}{\arabic{enumi}}
\renewcommand{\theenumii}{.\arabic{enumii}}
\renewcommand{\labelenumii}{\arabic{enumi}.\arabic{enumii}.}
\renewcommand{\theenumiii}{.\arabic{enumiii}}
\renewcommand{\labelenumiii}{\arabic{enumi}.\arabic{enumii}.\arabic{enumiii}.}
\usepackage{indentfirst}
\usepackage{fontspec}
\setmainfont{Times New Roman}

\begin{document}
\begin{titlepage}
\newpage



\begin{center}
\vfill
%\framepage

Федеральное государственное бюджетное образовательное учреждение высшего образования «МИРЭА — Российский технологический университет»\\
\ \\

\hfill\vbox
{
\hbox{Кафедра высшей математики}
}

\vfill

{\large\bf Разработка безопасного аудиокодека с шифрованием трафика\\}
\ \\
Реферат студентов 1 курса института искусственного интеллекта\\
Коровяковского Степана и Урвачева Романа

\vfill

%{
%\hbox{Научный руководитель:}
%\hbox to 16cm {к.т.н, доцент \hrulefill Х.З.~Мамаев}
%\hbox{}
%\hbox{Заведующий кафедрой:}
%\hbox to 16cm {д.ф.-м.н., профессор \hrulefill В.С.~Луговской}
%}

\vfill

Москва 2022
\end{center}

\end{titlepage}
\tableofcontents
\newpage
\chapter*{Введение}
\chapter{Изучение способов передачи аудиоданных в информационно-телекоммуникационных сетях}
\section{Телекоммуникационные технологии}
Каждому поколению свойственно разрабатывать новые технические средства, совершенствовать систему учета, обработки, передачи и хранения данных. Первыми телекоммуникационными средствами признан телеграф, телефон, телетайп, радиоприемник. Середина XIX столетия отмечена массовым использованием спутниковой связи, вычислительной техники, компьютерной сети. В результате это положительно отразилось на развитии новых телекоммуникационных технологий.
\par Современный мир невозможен без телекоммуникационных технологий, которые стирают государственные границы и расстояние между людьми, делают доступной мобильную и видеосвязь и позволяют решать множество задач в сфере управления, образования, коммерции. Каждый человек сталкивается с ними ежедневно, деля телефонные звонки, проверяя почту или покупая товары в интернет-магазинах.

\subsection{Определение и понятие телекоммуникационных \\ технологий}
Общее понятие информационных и коммуникационных технологий включает в себя совокупность методов, процессов и устройств, позволяющих получать, собирать, накапливать, хранить, обрабатывать и передавать информацию, закодированную в цифровом виде или существующую в аналоговом виде.
\par В более узком смысле под телекоммуникационными технологиями понимается совокупность программных и аппаратных средств, позволяющих устанавливать связь без использования проводов и передавать пакеты информации.

\subsection{Виды телекоммуникационных технологий}
Телекоммуникационные технологии могут быть рассмотрены как сервисы, предоставляемые провайдерами различного уровня.
По этому принципу можно выделить следующие виды телекоммуникационных технологий:
\begin{itemize}
\item телефонная связь, современная телефонная связь позволяет легко переключаться с аналогового стандарта на цифровой, подключать к интернет городские телефоны и соединять в одну сеть аналоговые и мобильные устройства;
\item радиосвязь, которая сегодня превратилась в сотовую связь, телефон, перемещаясь в пределах сети, оказывается в зоне действия различных передающих устройств;
\item спутниковая связь, которая используется провайдерами для создания систем мобильной связи и для государственных систем связи;
\item интернет – наиболее распространенный вид телекоммуникационных технологий, при которых подключение к сети может осуществляться как проводным, так и беспроводным способом.
\end{itemize}

\subsection{Основные типы информационно-телекоммуникационных сетей}
Телекоммуникационные технологии, используемые в интернете, сейчас переживают этап бурного развития и роста. С каждым днём создаётся всо больше и больше новых сетей различных типов, среди которых основными являются:
\begin{itemize}
\item локальные сети компаний или учреждений (Local Area Network - LAN), связь между компьютерами в них осуществляется и проводным и беспроводным способом, количество пользователей этих сетей ограничено. Локальные сети могут быть корпоративными, в некоторых странах создаются и городские локальные сети;
\item глобальные сети (Wide Area Network – WAN) представляют совокупность большого количества узлов-компьютеров, расположенных в разных странах мира и связанных между собой каналами оптово-волоконной связи. К этим сетям, представляющим услуги провайдеров, подключаются локальные сети.
\end{itemize}

\subsection{Технические и программные средства \\ телекоммуникационных технологий}
Работоспособность интернета основана на использовании сетевых узлов и каналов связи. К узлам относятся как отдельные компьютеры, так и хостинги, предоставляющие IP-адреса и доменные имена.
Каналы связи делятся на 4 типа:
\begin{itemize}
\item аналоговые телефонные сети;
\item провода, по которым передается электричество;
\item оптоволоконные каналы связи;
\item беспроводные каналы связи, модемные или спутниковые.
\end{itemize}
К телекоммуникационным каналам связи относятся, в основном, третий и четвертый типы.

Среди коммуникаций, используемых для организации связи, можно отдельно отметить программы, обеспечивающие работу телекоммуникационного оборудования такого, как:
\begin{itemize}
\item IP-АТС;
\item маршрутизаторы;
\item компьютеры.
\end{itemize}

\subsection{Основные задачи сетевых телекоммуникационных технологий}
Различные сетевые телекоммуникационные технологии позволяют решать такие задачи, как:
\begin{itemize}
\item передачу информации в необходимых форматах;
\item выстраивание коммуникаций;
\item обеспечение взаимодействия различных участников сети.
\end{itemize}

Среди новых технологий особое место занимают программы, позволяющие работать в режиме нетворкинга, объединение CRM-систем с возможностями социальных сетей и многое другое.

Создание корпоративных сетей как офисных, компьютерных, так и телефонных, также попадает в область сетевых технологий, призванных обеспечить синергию за счет эффективной коммуникации пользователей.

~

Спектр возможностей использования телекоммуникационных технологий расширяется с каждым днем. Сложно сказать, что именно будет предложено завтра в этой области, чтобы сделать связь доступнее, а производственные процессы – проще.

\section{Описание способов передачи аудиоданных}
\subsection{Передача аудиоданных по телефонным каналам связи}

Взаимное проникновение вычислительной техники и технических средств связи оказало серьезное влияние как на структуру компьютеров, так и на структуру каналов связи.

Средства связи, предназначенные для передачи информации между людьми, имеют длительную историю, развитую структуру (в мировом масштабе), мощную научную и технологическую базу и, начиная с 60-х годов, стали использоваться для передачи данных, т.е. для передачи информации между техническими средствами вычислительной техники, что потребовало включения в каналы связи дополнительных технических устройств.

В настоящее время для распределенных вычислительных систем наиболее широко используются телефонные каналы.

На рис. 1.2.1.1 представлена упрощенная схема линии аналоговой междугородней телефонной связи.

\includegraphics[scale=1.4]{66}
{\centering\caption{Рис. 1.2.1.1 Схема междугородней телефонной связи}\\}
~

На участке от телефонного аппарата до местной АТС происходит передача в первичной полосе частот « 200 - 3100 Гц (полоса частот человеческого голоса). При этом от каждого аппарата до АТС проводится двухпроводная электрическая линия для передачи этого сигнала, в дальнейшем происходит преобразование его в иную форму с целью уплотнения передачи. В каждом из последующих каналов идет очень большое количество передач. Существует два типа уплотнения: частотное и временное. В традиционных линиях связи, как правило, используется частотное уплотнение.

\subsubsection{Описание типов каналов связи}

В зависимости от типа передачи различают аналоговые (традиционно используемые, имеющие длительную историю развития) и цифровые каналы (систем ИКМ, ШОЫ и др.), являющиеся битовым трактом с цифровым импульсным сигналом на выходе и входе канала. Цифровые каналы отличаются рядом преимуществ перед аналоговыми, поэтому вновь создаваемые системы передачи данных стараются строить на основе цифровых каналов. Следует отметить, что цифровые каналы весьма успешно применяются не только для передачи данных, но и в средствах бытовой связи (звук, изображение и г.д.), при этом аналоговые сигналы кодируются в цифровые перед передачей в канал.

Термины «аналоговый» и «цифровой» соответствуют непрерывным и дискретным процессам и используются при обсуждении коммуникационных систем в различных контекстах - данных, сигналов и передачи.

Аналоговые данные представляются физической величиной, которая может изменяться в непрерывном диапазоне значений. Величина прямо пропорциональна данным или является их функцией.

Цифровые данные принимают дискретные значения - текст, целые числа, двоичные данные.

Аналоговый сигнал - непрерывно изменяющаяся электромагнитная волна, распространяющаяся в различных средах.

Цифровой сигнал - дискретный (разрывной) сигнал, такой, как последовательность импульсов напряжения.

Возможны четыре вида передачи данных:
\begin{enumerate}
\item) цифровые данные - цифровой сигнал, используется наиболее простое оборудование;
\item) аналоговые данные - цифровой сигнал, необходимо преобразование аналоговых данных в цифровую форму, что позволяет использовать современное (высокоэффективное) оборудование передачи данных;
\item) цифровые данные - аналоговый сигнал, необходимость преобразования связана с тем, что через некоторые среды (оптоволокно, беспроводные среды) может распространяться только аналоговый сигнал;
\item) аналоговые данные - аналоговый сигнал, традиционная передача, аналоговые данные легко преобразуются в аналоговый сигнал.
\end{enumerate}
\subsubsection{Основные преимущества цифровой передачи данных}
Среди преимуществ цифровой передачи необходимо отметить следующие.
\begin{itemize}
\item \qquad Быстрое развитие цифровых систем и уменьшение цены и размеров оборудования, цены и размеры аналогового оборудования остаются на прежнем уровне. Обслуживание цифровых систем намного дешевле аналоговых.
\item \qquad Использование повторителей (в цифровых системах) вместо аналоговых усилителей позволяет передавать данные на большие расстояния по менее качественным линиям (нет накопления шумов) - сохранение целостности данных.
\item \qquad Большая пропускная способность дает возможность более полно использовать пропускную способность оптоволокна и спутниковых средств связи. Временное разделение оказывается более эффективным, чем частотное.
\item \qquad Используется интеграция, когда при обработке аналоговой и цифровой информации по цифровым технологиям все сигналы имеют одинаковую форму (вид). Эго позволяет сэкономить на оборудовании и трудозатратах при интеграции: голос, видео, цифровые данные.
\end{itemize}
\subsubsection{Модемы}
Термин «модем» (ЭСЕ) применяется в настоящие время (в связи с распространением цифровых каналов) достаточно широко, при этом необязательно подразумевается какая-либо модуляция, а просто называются определенные операции преобразования сигналов, поступающих от ЭТЕ для их дальнейшей передачи по используемому каналу.

Существует очень много разновидностей модемов, отличающихся:
\begin{itemize}
\item по конструкции - внутренние (вставляемые в разъемы компьютера) и внешние, портативные, групповые и т.н.;
\item по методу передачи - асинхронные, синхронные, синхронноасинхронные.
\end{itemize}

Асинхронный метод передачи (или стартстопный) - посимвольный режим передачи с контролем начала и конца символа, имеет низкую скорость и малую эффективность. Синхронный метод передачи осуществляет объединение большого количества символов или байт в отдельные блоки - кадры, которые передаются без задержек между восьмибитными элементами.

~

\subsubsection{Режимы работы в зависимости от направления передачи}
Очень важной характеристикой канала передачи являются режимы его работы в зависимости от направления возможной передачи данных:
\begin{itemize}
\item симплексный - передача осуществляется по линии связи только в одном направлении;
\item полудуплексный - передача ведется в обоих направлениях, но попеременно во времени (технология Ethernet);
\item дуплексный - передача ведется одновременно в двух направлениях.
\end{itemize}

Дуплексный режим - наиболее универсальный и производительный. Самым простым вариантом организации дуплексного режима является использование двух независимых функциональных каналов (двух пар проводников или двух световодов) в кабеле, каждый из которых работает в симплексном режиме, т.е. передает данные в одном направлении. Такая организация дуплексного режима применяется во многих сетевых технологиях (Fast Ethernet, ATM и т.п.).

\subsection{Передача аудиоданных по радиосвязи}

Радиосвязь – быстрый и относительно надежный способ передачи данных на большие расстояния. При этом нет необходимости в использовании физического носителя, например проводов.

Свойства волн разной длины напрямую влияют на их применение для обеспечения радиосвязи. Кроме того, на качество передачи информации с их помощью влияют следующие факторы:

\begin{itemize}
\item высота приемной и передающей антенн;
\item рельеф поверхности;
\item солнечная активность, метеоусловия, время суток.
\end{itemize}

\subsubsection{Процесс приема-передачи информации}

Процесс приема-передачи информации с помощью радиоволн состоит из следующих основных этапов:

\begin{enumerate}

\item) формирование сигнала;
\item) выделение несущей частоты;
\item) связывание передаваемой информации с несущей частотой (модуляция);
\item) трансформация сигнала в дискретный вид, его кодирование (для цифровых систем);
\item) передача в радиоэфир с помощью антенны;
\item) прием сигнала;
\item) декодировка и демодуляция;
\item) преобразование сигнала в форму понятную абоненту.
\end{enumerate}

\subsubsection{Оборудование для осуществления передачи данных}
Чтобы реализовать обмен информации необходимо чтобы у принимающей и передающей стороны в наличии было следующее оборудование:

\begin{itemize}
\item передатчик;
\item антенна;
\item ретрансляционное устройство – позволяет увеличить дальность передачи сигнала;
\item принимающее устройство;
\item оборудование модуляции-демодуляции, сжатия, оцифровки и кодирования;
\item фильтры помех, усилители.
\end{itemize}

\subsubsection{Способы передачи данных по радиосвязи}
Применяется несколько способов радиосвязи, для каждого из которых используется специфическое оборудование. Три наиболее распространенных вида:
\begin{itemize}
\item сотовая связь;
\item радиорелейная связь;
\item спутниковая связь.
\end{itemize}


\subsubsection{Сферы применения радиосвязи}

Возможность практически мгновенной передачи информации на любые расстояния создает широкие возможности использования во всех сферах деятельности человека. Радиосвязь успешно применяется в следующих отраслях:

\begin{itemize}
\item телевизионное и радиовещание;
\item качественная связь по безопасным линиям востребована в военной отрасли. Позволяет осуществлять управление и координацию боевых подразделений;
\item в области транспорта – обеспечивается постоянная связь с поездами, морскими и речными судами, самолетами, грузовыми и легковыми автомобилям (полиция, скорая помощь, такси, курьерские службы);
\item организация диспетчерских служб;
\item обеспечение различных видов коммуникации: спутниковая, мобильная связь;
\item беспроводное подключение к сети Интернет.
\end{itemize}

Также широкие возможности коммуникации являются неотъемлемым инструментом практически любого современного бизнеса. При помощи беспроводной связи можно успешно решать вопросы управления удаленными объектами.


\subsection{Передача аудиоданных через системы спутниковой связи}

Системы спутниковой связи (ССС) широко используются во многих регионах мира и стали неотъемлемой частью инфраструктуры телекоммуникаций большинства стран. Не только промышленно развитые страны с разнообразными современными сетями телекоммуникаций, но все чаще и развивающиеся страны успешно внедряют ССС. 

Новые спутниковые приложения обеспечивают быстрое создание новых широковещательных служб и частных сетей.



\subsubsection{Устройство спутниковой связи}
Спутник - устройство связи, которое принимает сигналы от земной станции (ЗС), усиливает и транслирует в широковещательном режиме одновременно на все ЗС, находящиеся в зоне видимости спутника. Спутник не инициирует и не терминирует никакой пользовательской информации за исключением сигналов контроля и коррекции возникающих технических проблем и сигналов его позиционирования. Спутниковая передача начинается в некоторой ЗС, проходит через спутник, и заканчивается в одной или большем количестве ЗС.

ССС состоит из трех базисных частей: космического сегмента, сигнальной части и наземного сегмента (рис. 1.2.3.1). Космический сегмент охватывает вопросы проектирования спутника, расчета орбиты и запуска спутника. Сигнальная часть включает вопросы используемого спектра частоты, влияния расстояния на организацию и поддержание связи, источники интерференции сигнала, схем модуляции и протоколов передачи. Наземный сегмент включает размещение и конструкцию ЗС, типы антенн, используемых для различных приложений, схемы мультиплексирования, обеспечивающие эффективный доступ к каналам спутника.

\includegraphics[scale=0.7]{70}
{\centering\caption{\newline Рис. 1.2.3.1}\\}

~

На рис. 1.2.3.1: космический сегмент - спутники связи, сигнальная часть - наземная станция сопряжения, наземный сегмент - спутниковый телефон.

\subsubsection{Преимущества и ограничения систем спутниковой связи}
Системы спутниковой связи имеют уникальные особенности, отличающие их от других систем связи. Некоторые особенности обеспечивают преимущества, делающие спутниковую связь привлекательной для ряда приложений. Другие создают ограничения, которые неприемлемы при реализации некоторых прикладных задач.

ССС имеет ряд преимуществ:
\begin{itemize}
\item Устойчивые издержки. Стоимость передачи через спутник по одному соединению не зависит от расстояния между передающей и принимающей ЗС. Более того, все спутниковые сигналы - широковещательные. Стоимость спутниковой передачи, следовательно, остается неизменной независимо от числа принимающих ЗС.
\item Широкая полоса пропускания.
\item Малая вероятность ошибки. В связи с тем, что при цифровой спутниковой передаче побитовые ошибки весьма случайны, применяются эффективные и надежные статистические схемы их обнаружения и исправления.
\end{itemize}

Выделим также ряд ограничений в использовании ССС:

\begin{itemize}
\item Значительная задержка. Большое расстояние от ЗС до спутника на геосинхронной орбите приводит к задержке распространения, длиной почти в четверть секунды. Эта задержка вполне ощутима при телефонном соединении и делает чрезвычайно неэффективным использование спутниковых каналов при неадаптированной для ССС передаче данных.
\item Размеры ЗС. Крайне слабый на некоторых частотах спутниковый сигнал, доходящий до ЗС (особенно для спутников старых поколений), заставляет увеличивать диаметр антенны ЗС, усложняя тем самым процедуру размещения станции.
\item Защита от несанкционированного доступа к информации. Широковещание позволяет любой ЗС, настроенной на соответствующую частоту, принимать транслируемую спутником информацию. Лишь шифрование сигналов, зачастую достаточно сложное, обеспечивает защиту информации от несанкционированного доступа.
\item Интерференция. Спутниковые сигналы, действующие в Ku- или Ka-полосах частот (о них ниже), крайне чувствительны к плохой погоде. Спутниковые сети, действующие в C-полосе частот, восприимчивы к микроволновым сигналам. Интерференция вследствие плохой погоды ухудшает эффективность передачи в Ku- и Ka-полосах на период от нескольких минут до нескольких часов. Интерференция в С-полосе ограничивает развертывание ЗС в районах проживания с высокой концентрацией жителей.
\end{itemize}

Влияние упомянутых преимуществ и ограничений на выбор спутниковых систем для частных сетей довольно значительно. Решение об использовании ССС, а не распределенных наземных сетей, всякий раз необходимо экономически обосновать.


\subsubsection{Роль наземного сегмента}
Технологическое развитие привело к значительному уменьшению размеров земных сегментов. На начальном этапе спутник не превышал нескольких сотен килограммов, а ЗС представляли собой гигантские сооружения с антеннами более 30 м в диаметре. Современные спутники весят несколько тонн, а антенны, зачастую не превышающие 1 м в диаметре, могут быть установлены в самых разнообразных местах. Тенденция уменьшения размеров ЗС вместе с упрощением установки оборудования приводит к снижению его стоимости. На сегодняшний день стоимость ЗС является, пожалуй, главной характеристикой, определяющей широкое распространение ССС. Преимущество спутниковой связи основано на обслуживании географически удаленных пользователей без дополнительных расходов на промежуточное хранение и коммутацию. Любые факторы, понижающие стоимость установки новой ЗС, однозначно содействуют развитию приложений, ориентированных на использование ССС.

~

Последние достижения технологии в области спутниковой связи говорят о больших потенциальных возможностях ССС в расширении пропускной способности каналов передачи, разработке и внедрении новых служб связи. Будущее ССС за широкополосными широковещательными приложениями и спутниковыми системами подвижной связи.


\subsection{Передача аудиоданных через Интернет}
Аудио через IP — распространение цифрового аудио по IP-сети, такой как Интернет. Все чаще используется для обеспечения высококачественной передачи звука на большие расстояния.

\subsubsection{IP - телефония}

IP-телефония — телефонная связь по протоколу IP. Под IP-телефонией подразумевается набор коммуникационных протоколов, технологий и методов, обеспечивающих традиционные для телефонии набор номера, дозвон и двустороннее голосовое общение, а также видеообщение по сети Интернет или любым другим IP-сетям. Сигнал по каналу связи передаётся в цифровом виде и, как правило, перед передачей преобразовывается, чтобы удалить избыточность информации и снизить нагрузку на сеть передачи данных.

~

Самая главная особенность рассматриваемой связи — это то, что передача информации происходит не по специально выделенным телефонным линиям, а через компьютерную сеть. Естественно, этот нюанс диктует основные принципы работы технологии.

Полученный сигнал изначально необходимо оцифровать и перевести в форму, соответствующую требованиям протокола TCP-IP. Как правило, эта обязанность возлагается на специальные инструменты — VoIP-шлюзы или, в случае с АТС, DSP-процессор. Затем пакет передается абоненту, где данные переживают обратную трансформацию в понятный человеку вид (текст, картинка или звук).

Такой подход к организации связи позволяет получить сразу несколько преимуществ, ощутимых не только коммерческим клиентам, а и обычным пользователям:
\begin{itemize}
\item IP-телефония существенно снижает расходы на связь;
\item Качество передаваемой информации не страдает вне зависимости от расстояния;
\item Вариативность допускает легкую комбинацию несколько каналов;
\item Программное управление заметно расширяет функционал;
\item Оцифрованные данные можно закодировать.
\end{itemize}

К слову, одно из преимуществ рассматриваемого сервиса — это использование не физической, а виртуальной АТС.

~

АТС или автоматическая телефонная станция  — система устройств, обеспечивающая автоматическое (без участия оператора или телефонисток) соединение и поддержание телефонной связи между абонентами этой АТС.

\subsection{Интернет-радио}
Ещё одной разновидностью передачи аудиоданных по сети Интернет является интернет-радио.

Интернет-радио или веб-радио — группа технологий передачи потоковых аудиоданных через сеть Интернет для осуществления широковещательных передач. Также, в качестве термина интернет-радио или веб-радио может пониматься радиостанция, использующая для вещания технологию потокового вещания в глобальной сети Интернет.

\subsubsection{Принцип работы}
В технологической основе системы лежит три элемента:
\begin{itemize}
\item Станция — генерирует аудиопоток (либо из списка звуковых файлов, либо прямой оцифровкой с аудиокарты, либо копируя существующий в сети поток) и направляет его серверу. (Станция потребляет минимум трафика, потому что создаёт один поток)
\item Сервер (повторитель потока) — принимает аудиопоток от станции и перенаправляет его копии всем подключённым к серверу клиентам, по сути является репликатором данных. (Трафик сервера пропорционален количеству слушателей + 1)
\item Клиент — принимает аудиопоток от сервера и преобразует его в аудиосигнал, который и слышит слушатель интернет-радиостанции. Можно организовывать каскадные системы радиовещания, используя в качестве клиента повторитель потока (клиент, как и станция, потребляет минимум трафика; трафик клиента-сервера каскадной системы зависит от количества слушателей такого клиента.)
\end{itemize}

В качестве станции могут выступать обычная программа-аудиоплеер со специальным плагином-кодеком или специализированная программа (например — ICes, EzStream, SAM Broadcaster, RadioShure), а также аппаратное устройство, преобразующее аналоговый аудиопоток в цифровой.

Существует большое количество серверов интернет-вещания. Широко распространён сервер Shoutcast компании Nullsoft, разработанный специально для своего проигрывателя Winamp. Совместимый с Shoutcast сервер Icecast обладает гораздо большей функциональностью, распространяется свободно (на условиях GNU GPL) и бесплатно. В отличие от Shoutcast, Icecast способен передавать несколько аудиопотоков и требует меньше ресурсов на аудиопоток, чаще обновляется, поддерживает UTF-теги и разные форматы аудио, но он намного сложнее в настройке. Также сервера могут различаться по форматам аудиоданных, например: MP3, Ogg/Vorbis, RealAudio.

В качестве клиента можно использовать любой медиаплеер, поддерживающий потоковое аудио и способный декодировать формат, в котором вещает радио.

На рисунке 1.2.5.1 представлена упрощенная схема интернет-радио.

\includegraphics[scale=0.9]{internet_radio}
{\centering\caption{\newline Рис. 1.2.5.1 Схема работы интернет-радио}\\}



\newpage
\chapter{Изучение алгоритмов сжатия данных, в частности, сжатия
аудиоинформации.}
\section{Определение}
\textbf{Сжатие данных} — алгоритмическое обратимое преобразование данных, производимое с целью уменьшения занимаемого ими объёма. Применяется для более рационального использования устройств хранения и передачи данных.
\par Сжатие основано на устранении избыточности, содержащейся в исходных данных. Простейшим примером избыточности является повторение в тексте фрагментов (например, слов естественного или машинного языка). Подобная избыточность обычно устраняется заменой повторяющейся последовательности ссылкой на уже закодированный фрагмент с указанием его длины. Другой вид избыточности связан с тем, что некоторые значения в сжимаемых данных встречаются чаще других. Сокращение объёма данных достигается за счёт замены часто встречающихся данных короткими кодовыми словами, а редких — длинными (энтропийное кодирование). Сжатие данных, не обладающих свойством избыточности (например, случайный сигнал или белый шум, зашифрованные сообщения), принципиально невозможно без потерь.
\section{Принципы сжатия данных}
В основе любого способа сжатия лежит модель источника данных, или, точнее, модель избыточности. Иными словами, для сжатия данных используются некоторые априорные сведения о том, какого рода данные сжимаются. Не обладая такими сведениями об источнике, невозможно сделать никаких предположений о преобразовании, которое позволило бы уменьшить объём сообщения. Модель избыточности может быть статической, неизменной для всего сжимаемого сообщения, либо строиться или параметризоваться на этапе сжатия (и восстановления). Методы, позволяющие на основе входных данных изменять модель избыточности информации, называются адаптивными. Неадаптивными являются обычно узкоспециализированные алгоритмы, применяемые для работы с данными, обладающими хорошо определёнными и неизменными характеристиками. Подавляющая часть достаточно универсальных алгоритмов является в той или иной мере адаптивной.

Все методы сжатия данных делятся на два основных класса:
\begin{itemize}
\item Сжатие без потерь
\item Сжатие с потерями
\end{itemize}
При использовании сжатия без потерь возможно полное восстановление исходных данных, сжатие с потерями позволяет восстановить данные с искажениями, обычно несущественными с точки зрения дальнейшего использования восстановленных данных. Сжатие без потерь обычно используется для передачи и хранения текстовых данных, компьютерных программ, реже — для сокращения объёма аудио- и видеоданных, цифровых фотографий и т. п., в случаях, когда искажения недопустимы или нежелательны. Сжатие с потерями, обладающее значительно большей, чем сжатие без потерь, эффективностью, обычно применяется для сокращения объёма аудио- и видеоданных и цифровых фотографий в тех случаях, когда такое сокращение является приоритетным, а полное соответствие исходных и восстановленных данных не требуется. 
\section{Основные характеристики алгоритмов сжатия}
\subsection{Коэффициент сжатия}
\textbf{Коэффициент сжатия} — основная характеристика алгоритма сжатия. Она определяется как отношение объёма исходных несжатых данных к объёму сжатых данных, то есть: $k = \frac{S_o}{S_c}$, где $k$ - коэффициент сжатия, $S_o$ - объём исходных данных, а $S_c$ - объём сжатых. Таким образом, чем выше коэффициент сжатия, тем алгоритм эффективнее. Следует отметить:
\begin{itemize}
\item если k = 1, то алгоритм не производит сжатия, то есть выходное сообщение оказывается по объёму равным входному;
\item если k < 1, то алгоритм порождает сообщение большего размера, нежели несжатое, то есть, совершает «вредную» работу.
\end{itemize}
\subsection{Допустимость потерь}
Основным критерием различия между алгоритмами сжатия является описанное выше наличие или отсутствие потерь. В общем случае алгоритмы сжатия без потерь универсальны в том смысле, что их применение безусловно возможно для данных любого типа, в то время как возможность применения сжатия с потерями должна быть обоснована. Для некоторых типов данных искажения не допустимы в принципе. В их числе:
\begin{itemize}
\item символические данные, изменение которых неминуемо приводит к изменению их семантики: программы и их исходные тексты, двоичные массивы и т.п.;
\item жизненно важные данные, изменения в которых могут привести к критическим ошибкам: например, получаемые с медицинской измерительной аппаратуры или контрольных приборов летательных, космических аппаратов и т.п.;
\item многократно подвергаемые сжатию и восстановлению промежуточные данные при многоэтапной обработке графических, звуковых и видеоданных.
\end{itemize}
\subsection{Системные требования алгоритмов}
Различные алгоритмы могут требовать различного количества ресурсов вычислительной системы, на которых они реализованы:
\begin{itemize}
\item оперативной памяти (под промежуточные данные);
\item постоянной памяти (под код программы и константы);
\item процессорного времени.

\end{itemize}
В целом, эти требования зависят от сложности и «интеллектуальности» алгоритма. Общая тенденция такова: чем эффективнее и универсальнее алгоритм, тем большие требования к вычислительным ресурсам он предъявляет. Тем не менее, в специфических случаях простые и компактные алгоритмы могут работать не хуже сложных и универсальных. Системные требования определяют их потребительские качества: чем менее требователен алгоритм, тем на более простой, а следовательно, компактной, надёжной и дешёвой системе он может быть реализован.

Так как алгоритмы сжатия и восстановления работают в паре, имеет значение соотношение системных требований к ним. Нередко можно, усложнив один алгоритм, значительно упростить другой. Таким образом, возможны три варианта:
\subsubsection{Алгоритм сжатия требует больших вычислительных ресурсов, нежели алгоритм восстановления.}
Это наиболее распространённое соотношение, характерное для случаев, когда однократно сжатые данные будут использоваться многократно. В качестве примера можно привести цифровые аудио- и видеопроигрыватели.
\subsubsection{Алгоритмы сжатия и восстановления требуют приблизительно равных вычислительных ресурсов.}
Наиболее приемлемый вариант для линий связи, когда сжатие и восстановление происходит однократно на двух её концах (например, в цифровой телефонии).
\subsubsection{Алгоритм сжатия существенно менее требователен, чем алгоритм восстановления.}
Такая ситуация характерна для случаев, когда процедура сжатия реализуется простым, часто портативным, устройством, для которого объём доступных ресурсов весьма критичен, например, космический аппарат или большая распределённая сеть датчиков. Это могут быть также данные, распаковка которых требуется в очень малом проценте случаев, например запись камер видеонаблюдения.
\section{Кодирование аудиоданных}
\subsection{Принципы оцифровки звука}
\textbf{Цифровой звук} — это аналоговый звуковой сигнал, представленный посредством дискретных численных значений его амплитуды.

\textbf{Оцифровка звука} — технология осуществления замеров амплитуды звукового сигнала с определенным временным шагом и последующей записи полученных значений в численном виде.
Другое название оцифровки звука — аналогово-цифровое преобразование звука.

Оцифровка звука включает в себя два процесса:
\begin{itemize}
\item процесс дискретизации (осуществление выборки) сигнала по времени
\item процесс квантования по амплитуде.
\end{itemize}

~

\includegraphics[scale=0.8]{Analog_to_digi}
{\centering\par{Рис. 1.4.1.1 Схема оцифровки звука}\\}

~
\subsubsection{Процесс дискретизации по времени}
\textbf{Процесс дискретизации по времени} — процесс получения значений сигнала, который преобразуется с определенным временным шагом — шагом дискретизации . Количество замеров величины сигнала, осуществляемых в единицу времени, называют частотой дискретизации или частотой выборки. Чем меньше шаг дискретизации, тем выше частота дискретизации и тем более точное представление о сигнале будет получено.
Это подтверждается теоремой Котельникова. Согласно ей, аналоговый сигнал с ограниченным спектром точно описуем дискретной последовательностью значений его амплитуды, если эти значения берутся с частотой, как минимум вдвое превышающей наивысшую частоту спектра сигнала. То есть, аналоговый сигнал, в котором находится частота спектра равная $F_m$, может быть точно представлен последовательностью дискретных значений амплитуды, если для частоты дискретизации $F_d$ выполняется: $F_d>2F_m$.
\par На практике это означает, что для того, чтобы оцифрованный сигнал содержал информацию о всем диапазоне слышимых частот исходного аналогового сигнала (20 Гц — 20 кГц) необходимо, чтобы выбранное значение частоты дискретизации составляло не менее 40 кГц.
\par Основная трудность оцифровки заключается в невозможности записать измеренные значения сигнала с идеальной точностью.
\subsubsection{Линейное квантование амплитуды}
Отведём для записи одного значения амплитуды сигнала в памяти компьютера $N$ бит. Значит, с помощью одного $N$-битного слова можно описать $2N$ разных положений. Пусть амплитуда оцифровываемого сигнала колеблется в пределах от −1 до 1 некоторых условных единиц. Представим этот диапазон изменения амплитуды — \textbf{динамический диапазон сигнала} — в виде $2N−1$ равных промежутков, разделив его на $2N$ уровней — \textbf{квантов}. Теперь, для записи каждого отдельного значения амплитуды, его необходимо округлить до ближайшего уровня квантования. Этот процесс носит название \textbf{квантования по амплитуде}.

\textbf{Квантование по амплитуде} — процесс замены реальных значений амплитуды сигнала значениями, приближенными с некоторой точностью. Каждый из $2^N$ возможных уровней называется \textbf{уровнем квантования}, а расстояние между двумя ближайшими уровнями квантования называется \textbf{шагом квантования}. Если амплитудная шкала разбита на уровни линейно, квантование называют линейным (однородным).
Точность округления зависит от выбранного количества $2N$ уровней квантования, которое, в свою очередь, зависит от количества бит $N$, отведенных для записи значения амплитуды. Число $N$ называют \textbf{разрядностью квантования}, а полученные в результате округления значений амплитуды числа — \textbf{отсчетами} или \textbf{семплами}. Принимается, что погрешности квантования, являющиеся результатом квантования с разрядностью 16 бит, остаются для слушателя почти незаметными. Этот способ оцифровки сигнала — дискретизация сигнала во времени в совокупности с методом однородного квантования — называется \textbf{импульсно-кодовой модуляцией, ИКМ (англ. Pulse Code Modulation — PCM)}.

Оцифрованный сигнал в виде набора последовательных значений амплитуды уже можно сохранить в памяти компьютера. В случае, когда записываются абсолютные значения амплитуды, такой формат записи называется PCM (Pulse Code Modulation). Стандартный аудио компакт-диск (CD-DA), применяющийся с начала 80-х годов, хранит информацию в формате PCM с частотой дискретизации 44.1 кГц и разрядностью квантования 16 бит.  
\subsubsection{Аналогово-Цифровые Преобразователи}
Вышеописанный процесс оцифровки звука выполняется
\textbf{аналогово-цифровыми преобразователями (АЦП)}.
Это преобразование включает в себя следующие операции:
\begin{enumerate}
\item \textbf{Ограничение полосы частот} производится при помощи фильтра нижних частот для подавления спектральных компонент, частота которых превышает половину частоты дискретизации.
\item \textbf{Дискретизация по времени}
\item \textbf{Квантование по уровню}
\item \textbf{Кодирование или оцифровка}, в результате которой значение каждого квантованного отсчета представляется в виде числа, соответствующего порядковому номеру уровня квантования.
\end{enumerate}

\subsection{Кодирование оцифрованного звука}
Для хранения цифрового звука существует много различных способов. Оцифрованный звук являет собой набор значений амплитуды сигнала, взятых через определенные промежутки времени.
\begin{itemize}
\item Блок оцифрованной аудио информации можно записать в файл без изменений, то есть последовательностью чисел — значений амплитуды. В этом случае существуют два способа хранения информации.
\begin{itemize}
\item Первый — PCM (Pulse Code Modulation — импульсно-кодовая модуляция) — способ цифрового кодирования сигнала при помощи записи абсолютных значений амплитуд. (В таком виде записаны данные на всех аудио CD.)
\item Второй — ADPCM (Adaptive Delta PCM — адаптивная относительная импульсно-кодовая модуляция) — запись значений сигнала не в абсолютных, а в относительных изменениях амплитуд (приращениях).
\end{itemize}
\item Можно сжать данные так, чтобы они занимали меньший объем памяти, нежели в исходном состоянии. Тут тоже есть два способа.
\begin{itemize}


\item Кодирование данных без потерь (lossless coding) — способ кодирования аудио, который позволяет осуществлять стопроцентное восстановление данных из сжатого потока. К нему прибегают в тех случаях, когда сохранение оригинального качества данных особо значимо. Существующие сегодня алгоритмы кодирования без потерь (например, Monkeys Audio) позволяют сократить занимаемый данными объем на 20-50\%, но при этом обеспечить стопроцентное восстановление оригинальных данных из полученных после сжатия.
\item Кодирование данных с потерями (lossy coding). Здесь цель — добиться схожести звучания восстановленного сигнала с оригиналом при как можно меньшем размере сжатого файла. Это достигается путём использования алгоритмов, «упрощающих» оригинальный сигнал (удаляющих из него «несущественные», неразличимые на слух детали). Это приводит к тому, что декодированный сигнал перестает быть идентичным оригиналу, а является лишь «похоже звучащим». Методов сжатия, а также программ, реализующих эти методы, существует много. Наиболее известными являются MPEG-1 Layer I,II,III (последним является всем известный MP3), MPEG-2 AAC (advanced audio coding), Ogg Vorbis, Windows Media Audio (WMA), TwinVQ (VQF), MPEGPlus, TAC, и прочие. В среднем, коэффициент сжатия, обеспечиваемый такими кодерами, находится в пределах 10-14 (раз). В основе всех lossy-кодеров лежит использование так называемой психоакустической модели. Она занимается этим самым «упрощением» оригинального сигнала. Степень сжатия оригинального сигнала зависит от степени его «упрощения» — сильное сжатие достигается путём «воинственного упрощения» (когда кодером игнорируются множественные нюансы). Такое сжатие приводит к сильной потере качества, поскольку удалению могут подлежать не только незаметные, но и значимые детали звучания.
\end{itemize}
\end{itemize}

\subsection{Сжатие аудиоданных без потерь}
Сокращение статистической избыточности основано на учёте свойств самих звуковых сигналов. Она определяется наличием корреляционной связи между соседними отсчетами цифрового звукового сигнала, устранение которой позволяет сокращать объём передаваемых данных на 15-25 \% по сравнению с их исходной величиной. Для передачи сигнала необходимо получить более компактное его представление, что возможно осуществить с помощью ортогонального преобразования. Важными условиями применения такого метода преобразования являются:

\begin{itemize}
\item возможность восстанавливать исходный сигнал без искажений
\item способность обеспечивать наибольшую концентрацию энергии в небольшом числе коэффициентов преобразования
\item быстрый вычислительный алгоритм
\end{itemize}
    
Этим требованиям отвечает модифицированное дискретно-косинусное преобразование (МДКП).

Уменьшить скорость цифрового потока позволяют методы кодирования, учитывающие статистику звуковых сигналов, например, вероятности появления уровней разной величины. Одним из таких методов является код Хаффмана, где наиболее вероятным значениям сигнала приписываются более короткие кодовые слова, а значения отсчетов, вероятность появления которых мала, кодируются кодовыми словами большей длины. Именно в силу этих двух причин в наиболее эффективных алгоритмах компрессии цифровых аудиоданных кодированию подвергаются не сами отсчеты звукового сигнала, а коэффициенты МДКП.

Подобные методы применяются при архивации файлов. 


\subsection{Сжатие аудиоданных с потерями}
Сжатие аудиоданных с потерями основывается на несовершенстве человеческого слуха при восприятии звуковой информации. Неспособность человека в определённых случаях различать тихие звуки в присутствии более громких, называемая эффектом маскировки, была использована в алгоритмах сокращения психоакустической избыточности. Эффекты слухового маскирования зависят от спектральных и временных характеристик маскируемого и маскирующего сигналов и могут быть разделены на две основные группы:
\begin{itemize}
\item частотное (одновременное) маскирование
\item временное (неодновременное) маскирование
\end{itemize}
Эффект маскирования в частотной области связан с тем, что в присутствии больших звуковых амплитуд человеческое ухо нечувствительно к малым амплитудам близких частот. То есть, когда два сигнала одновременно находятся в ограниченной частотной области, то более слабый сигнал становится неслышимым на фоне более сильного.

Маскирование во временной области характеризует динамические свойства слуха, показывая изменение во времени относительного порога слышимости (порог слышимости одного сигнала в присутствии другого), когда маскирующий и маскируемый сигналы звучат не одновременно. При этом следует различать явления послемаскировки (изменение порога слышимости после сигнала высокого уровня) и предмаскировки (изменение порога слышимости перед приходом сигнала максимального уровня). Более слабый сигнал становится неслышимым за 5 − 20 мс до включения сигнала маскирования и становится слышимым через 50 − 200 мс после его включения.

Наилучшим методом кодирования звука, учитывающим эффект маскирования, оказывается полосное кодирование. Сущность его заключается в следующем. Группа отсчетов входного звукового сигнала, называемая кадром, поступает на блок фильтров который разделяет сигнал на частотные поддиапазоны. На выходе каждого фильтра оказывается та часть входного сигнала, которая попадает в полосу пропускания данного фильтра. Далее, в каждой полосе с помощью психоакустической модели, анализируется спектральный состав сигнала и оценивается, какую часть сигнала следует передавать без сокращений, а какая лежит ниже порога маскирования и может быть переквантована на меньшее число бит. Для сокращения максимального динамического диапазона определяется максимальный отсчет в кадре и вычисляется масштабирующий множитель, который приводит этот отсчет к верхнему уровню квантования. Эта операция аналогична компандированию в аналоговом вещании. На этот же множитель умножаются и все остальные отсчеты. Масштабирующий множитель передается к декодеру вместе с кодированными данными для коррекции коэффициента передачи последнего. После масштабирования производится оценка порога маскирования и осуществляется перераспределение общего числа битов между всеми полосами.

Очевидно, что после устранения психоакустической избыточности звуковых сигналов их точное восстановления при декодировании оказывается уже невозможным. Методами устранения психофизической избыточности можно обеспечить сжатие цифровых аудиоданных в 10 − 12 раз без существенных потерь в качестве.

\subsection{Структура кодера сжатия аудиоданный с потерями}
\begin{itemize}
\item Исходный цифровой звуковой сигнал разделяется на частотные поддиапазоны и сегментируется по времени в блоке временной и частотной сегментации.
\item Длина кодируемой выборки зависит от формы временной функции звукового сигнала. При отсутствии резких выбросов по амплитуде используется так называемая длинная выборка, обеспечивающая высокое разрешение по частоте. В случае же резких изменений амплитуды сигнала длина кодируемой выборки резко уменьшается, что дает более высокое разрешение по времени. Решение об изменении длины кодируемой выборки принимает блок психоакустического анализа, вычисляя значение психоакустической энтропии сигнала.
\item После сегментации сигналы частотных поддиапазонов нормируются, квантуются и кодируются. В наиболее эффективных алгоритмах компрессии кодированию подвергаются не сами отсчеты выборки звукового сигнала, а соответствующие им коэффициенты МДКП.
\item Учёт закономерностей слухового восприятия звукового сигнала выполняется в блоке психоакустического анализа. Здесь по специальной процедуре для каждого частотного поддиапазона рассчитывается максимально допустимый уровень искажений (шумов) квантования, при котором они ещё маскируются полезным сигналом данного поддиапазона.
\item Блок динамического распределения бит в соответствии с требованиями психоакустической модели для каждого поддиапазона кодирования выделяет такое минимально возможное их количество, при котором уровень искажений, вызванных квантованием, не превышал порога их слышимости, рассчитанного психоакустической моделью.
\item Также могут использоваться:
\begin{itemize}
\item матрицирование стерео — сложение и вычитание левого и правого канала для устранения повторяющейся информации
\item специальные процедуры итерационных циклов, позволяющие управлять величиной энергии искажений квантования в поддиапазонах при недостаточном числе доступных для кодирования бит
\item процедуры линейного и обратного адаптивного предсказаний
\item техника сглаживания переходных шумов во временной области (Temporal Noise Shaping — TNS), позволяющая управлять микроструктурой искажений квантования внутри каждого поддиапазона кодирования
\end{itemize}
\end{itemize}
Многие другие приёмы могут послужить способом сократить объём данных звуковой информации. Даже простое сужение полосы частот сигнала вместе с уменьшением динамического диапазона может уже называться сжатием аудиоданных. Например, в стандарте сжатия звука в сотовой связи используется и то и другое. Стремясь удалить избыточность из звука, кодек при плохом качестве сигнала становится избирателен к определённым словам, упорно проглатывая их. 



















\end{document}





