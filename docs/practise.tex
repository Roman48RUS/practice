\babel@toc {russian}{}\relax 
\contentsline {chapter}{\numberline {1}Изучение способов передачи аудиоданных в \\ информационно-телекоммуникационных сетях}{3}{}%
\contentsline {section}{\numberline {1.1}Телекоммуникационные технологии}{3}{}%
\contentsline {subsection}{\numberline {1.1.1}Определение и понятие телекоммуникационных \\ технологий}{3}{}%
\contentsline {subsection}{\numberline {1.1.2}Виды телекоммуникационных технологий}{4}{}%
\contentsline {subsection}{\numberline {1.1.3}Основные типы информационно-телекоммуникационных сетей}{4}{}%
\contentsline {subsection}{\numberline {1.1.4}Технические и программные средства \\ телекоммуникационных технологий}{4}{}%
\contentsline {subsection}{\numberline {1.1.5}Основные задачи сетевых телекоммуникационных \\ технологий}{5}{}%
\contentsline {section}{\numberline {1.2}Описание способов передачи аудиоданных}{5}{}%
\contentsline {subsection}{\numberline {1.2.1}Передача аудиоданных по телефонным каналам связи}{5}{}%
\contentsline {subsubsection}{Описание типов каналов связи}{6}{}%
\contentsline {subsubsection}{Основные преимущества цифровой передачи данных}{7}{}%
\contentsline {subsubsection}{Модемы}{8}{}%
\contentsline {subsubsection}{Режимы работы в зависимости от направления передачи}{8}{}%
\contentsline {subsection}{\numberline {1.2.2}Передача аудиоданных по радиосвязи}{9}{}%
\contentsline {subsubsection}{Процесс приема-передачи информации}{9}{}%
\contentsline {subsubsection}{Оборудование для осуществления передачи данных}{9}{}%
\contentsline {subsubsection}{Способы передачи данных по радиосвязи}{10}{}%
\contentsline {subsubsection}{Сферы применения радиосвязи}{10}{}%
\contentsline {subsection}{\numberline {1.2.3}Передача аудиоданных через системы спутниковой связи}{10}{}%
\contentsline {subsubsection}{Устройство спутниковой связи}{11}{}%
\contentsline {subsubsection}{Преимущества и ограничения систем спутниковой связи}{11}{}%
\contentsline {subsubsection}{Роль наземного сегмента}{13}{}%
\contentsline {subsection}{\numberline {1.2.4}Передача аудиоданных через Интернет}{13}{}%
\contentsline {subsubsection}{IP - телефония}{13}{}%
\contentsline {subsection}{\numberline {1.2.5}Интернет-радио}{14}{}%
\contentsline {subsubsection}{Принцип работы}{14}{}%
\contentsline {chapter}{\numberline {2}Изучение алгоритмов сжатия данных, в частности, сжатия \\ аудиоинформации.}{16}{}%
\contentsline {section}{\numberline {2.1}Определение}{16}{}%
\contentsline {section}{\numberline {2.2}Принципы сжатия данных}{16}{}%
\contentsline {section}{\numberline {2.3}Основные характеристики алгоритмов сжатия}{17}{}%
\contentsline {subsection}{\numberline {2.3.1}Коэффициент сжатия}{17}{}%
\contentsline {subsection}{\numberline {2.3.2}Допустимость потерь}{17}{}%
\contentsline {subsection}{\numberline {2.3.3}Системные требования алгоритмов}{18}{}%
\contentsline {subsubsection}{Алгоритм сжатия требует больших вычислительных ресурсов, нежели алгоритм восстановления.}{18}{}%
\contentsline {subsubsection}{Алгоритмы сжатия и восстановления требуют приблизительно равных вычислительных ресурсов.}{18}{}%
\contentsline {subsubsection}{Алгоритм сжатия существенно менее требователен, чем алгоритм восстановления.}{19}{}%
\contentsline {section}{\numberline {2.4}Кодирование аудиоданных}{19}{}%
\contentsline {subsection}{\numberline {2.4.1}Принципы оцифровки звука}{19}{}%
\contentsline {subsubsection}{Процесс дискретизации по времени}{20}{}%
\contentsline {subsubsection}{Линейное квантование амплитуды}{20}{}%
\contentsline {subsubsection}{Аналогово-Цифровые Преобразователи}{21}{}%
\contentsline {subsection}{\numberline {2.4.2}Кодирование оцифрованного звука}{21}{}%
\contentsline {subsection}{\numberline {2.4.3}Сжатие аудиоданных без потерь}{22}{}%
\contentsline {subsection}{\numberline {2.4.4}Сжатие аудиоданных с потерями}{23}{}%
\contentsline {subsection}{\numberline {2.4.5}Структура кодера сжатия аудиоданный с потерями}{24}{}%
\contentsline {chapter}{\numberline {3}Изучение алгоритмов блочного и поточного шифрования данных}{26}{}%
\contentsline {section}{\numberline {3.1}Алгоритмы блочного шифрования данных}{26}{}%
\contentsline {subsection}{\numberline {3.1.1}Определение и основные свойства блочного шифра}{26}{}%
\contentsline {subsubsection}{Структура блочного шифра}{27}{}%
\contentsline {subsection}{\numberline {3.1.2}Алгоритм шифрования DES}{27}{}%
\contentsline {subsubsection}{Принцип работы DES}{27}{}%
\contentsline {subsubsection}{Процесс зашифрования}{28}{}%
\contentsline {subsubsection}{Генерирование ключей}{29}{}%
\contentsline {subsubsection}{Процесс расшифрования}{29}{}%
\contentsline {subsubsection}{Режимы использования DES}{30}{}%
\contentsline {subsubsection}{Достоинства и недостатки DES}{31}{}%
\contentsline {subsection}{\numberline {3.1.3}Алгоритм шифрования IDEA}{32}{}%
\contentsline {subsubsection}{Принцип работы}{32}{}%
\contentsline {subsubsection}{Шифрование}{33}{}%
\contentsline {subsubsection}{Расшифровка}{35}{}%
\contentsline {subsubsection}{Достоинства и недостатки IDEA}{35}{}%
\contentsline {subsection}{\numberline {3.1.4}Алгоритм шифрования RC5}{35}{}%
\contentsline {subsubsection}{Параметры алгоритма}{36}{}%
\contentsline {subsubsection}{Расширение ключа}{36}{}%
\contentsline {subsubsection}{Шифрование}{37}{}%
\contentsline {subsubsection}{Расшифрование}{38}{}%
\contentsline {section}{\numberline {3.2}Алгоритмы поточного шифрования данных}{38}{}%
\contentsline {subsection}{\numberline {3.2.1}Определение и основные свойства поточного шифра}{38}{}%
\contentsline {subsubsection}{Классификация потоковых шифров}{39}{}%
\contentsline {subsection}{\numberline {3.2.2}Алгоритм шифрования RC4}{40}{}%
\contentsline {subsubsection}{Принцип работы}{40}{}%
\contentsline {subsection}{\numberline {3.2.3}Алгоритм шифрования VMPC}{42}{}%
\contentsline {subsubsection}{Приницип работы}{42}{}%
\contentsline {subsubsection}{Особенности алгоритма VMPC}{43}{}%
\contentsline {subsection}{\numberline {3.2.4}Алгоритм шифрования А5}{43}{}%
\contentsline {subsubsection}{Принцип работы}{44}{}%
